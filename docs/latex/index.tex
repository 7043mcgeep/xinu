\section*{Embedded Xinu}

Embedded Xinu, Copyright (C) 2008, 2009, 2010, 2019. All rights reserved.

Version\-: 3.\-0


\begin{DoxyEnumerate}
\item What is Embedded Xinu?
\item Directory Structure
\item Prerequisites
\item Installation Instructions
\begin{DoxyEnumerate}
\item Build Embedded Xinu
\item Make serial and network connections
\item Enter Common Firmware Environment prompt
\item Set I\-P address
\item Load image over T\-F\-T\-P
\end{DoxyEnumerate}
\item Supported Platforms
\item Links
\end{DoxyEnumerate}

\subsection*{1. What is Embedded Xinu?}

Xinu (\char`\"{}\-Xinu is not unix\char`\"{}, a recursive acronym) is a U\-N\-I\-X-\/like operating system originally developed by Douglas Comer for instructional purposes at Purdue University in the 1980s.

Embedded Xinu is a reimplementation of the original Xinu operating system on the M\-I\-P\-S processor which is able to run on inexpensive wireless routers and is suitable for courses and research in the areas of Operating Systems, Hardware Systems, Embedded Systems, Networking, and Compilers.

\subsection*{2. Directory Structure}

Once you have downloaded and extracted the xinu tarball, you will see a basic directory structure\-: \begin{DoxyVerb}AUTHORS   device/   lib/     mem/      loader/   README  system/
compile/  include/  LICENSE  mailbox/  network/  shell/  test/
\end{DoxyVerb}



\begin{DoxyItemize}
\item {\ttfamily A\-U\-T\-H\-O\-R\-S} a brief history of contributors to the Xinu operating system in it's varying iterations.
\item {\ttfamily compile/} contains the Makefile and other necessities for building the Xinu system once you have a cross-\/compiler.
\item {\ttfamily device/} contains directories with source for all device drivers in Xinu.
\item {\ttfamily include/} contains all the header files used by Xinu.
\item {\ttfamily lib/} contains library folders (e.\-g., {\ttfamily libxc/}) with a Makefile and source for the library
\item {\ttfamily L\-I\-C\-E\-N\-S\-E} the license under which this project falls.
\item {\ttfamily loader/} contains assembly files and is where the bootloader will begin execution of O/\-S code.
\item {\ttfamily mailbox/} contains source for the mailbox message queuing system.
\item {\ttfamily mem/} contains source for page-\/based memory protection.
\item {\ttfamily network/} contains code for the T\-C\-P/\-I\-P networking stack.
\item {\ttfamily R\-E\-A\-D\-M\-E} this document.
\item {\ttfamily R\-E\-L\-E\-A\-S\-E} release notes for the current version.
\item {\ttfamily shell/} contains the source for all Xinu shell related functions.
\item {\ttfamily system/} contains the source for all Xinu system functions such as the nulluser process ({\ttfamily \hyperlink{initialize_8c}{initialize.\-c}}) as well as code to set up a C environment ({\ttfamily startup.\-S}).
\item {\ttfamily test/} contains a number of testcases (which can be run using the shell command {\ttfamily testsuite}).
\end{DoxyItemize}

\subsection*{3. Prerequisites}

\subsubsection*{3.\-1 Supported platform with hardware modification}

To run Embedded Xinu you need a supported router or virtual machine. Currently, Embedded Xinu supports Linksys W\-R\-T54\-G\-L, Linksys W\-R\-T160\-N\-L, and the Qemu-\/mipsel virtual machine. For an updated list of supported platforms, visit\-:

\href{http://xinu.mscs.mu.edu/List_of_supported_platforms}{\tt http\-://xinu.\-mscs.\-mu.\-edu/\-List\-\_\-of\-\_\-supported\-\_\-platforms}

In order to communicate with the router, you need to perform a hardware modification that will expose the serial port that exists on the P\-C\-B. For more information on this process, see\-:

\href{http://xinu.mscs.mu.edu/Modify_the_Linksys_hardware}{\tt http\-://xinu.\-mscs.\-mu.\-edu/\-Modify\-\_\-the\-\_\-\-Linksys\-\_\-hardware}

\subsubsection*{3.\-2 Cross-\/compiler}

To build Embedded Xinu you will need a cross-\/compiler from your host computer's architecture to M\-I\-P\-S\-E\-L (little endian M\-I\-P\-S for the 54\-G\-L router) or M\-I\-P\-S (big endian for the 160\-N\-L router). Instructions on how to do this can be found here\-:

\href{http://xinu.mscs.mu.edu/Build_Xinu#Cross-Compiler}{\tt http\-://xinu.\-mscs.\-mu.\-edu/\-Build\-\_\-\-Xinu\#\-Cross-\/\-Compiler}

\subsubsection*{3.\-3 Serial communication software}

Any serial communication software will do. The Xinu Console Tools include a program called tty-\/connect which can serve the purpose for a U\-N\-I\-X environment. More information about the Xinu Console Tools can be found at\-:

\href{http://xinu.mscs.mu.edu/Console_Tools#Xinu_Console_Tools}{\tt http\-://xinu.\-mscs.\-mu.\-edu/\-Console\-\_\-\-Tools\#\-Xinu\-\_\-\-Console\-\_\-\-Tools}

\subsubsection*{3.\-4 T\-F\-T\-P server software}

A T\-F\-T\-P server will provide the router with the ability to download and run the compiled Embedded Xinu image.

\subsection*{4. Installation Instructions}

\subsubsection*{4.\-1 Build Embedded Xinu}

Update the {\ttfamily M\-I\-P\-S\-\_\-\-R\-O\-O\-T} and {\ttfamily M\-I\-P\-S\-\_\-\-P\-R\-E\-F\-I\-X} variables in compile/mips\-Vars to correctly point to the cross-\/compiler on your machine.

Then, from the compile directory, simply run make, which should leave you with a xinu.\-boot file. This is the binary image you need to transfer to your router for it to run Embedded Xinu. The default build is for the W\-R\-T54\-G\-L router; change the compile/\-Makefile {\ttfamily P\-L\-A\-T\-F\-O\-R\-M} variable for other builds. See the compile/platforms directory for supported configurations.

\subsubsection*{4.\-2 Make serial and network connections}

After creating the {\ttfamily xinu.\-boot} image you can connect the router's serial port to your computer and open up a connection using the following settings\-:


\begin{DoxyItemize}
\item 8 data bits, no parity bit, and 1 stop bit (8\-N1)
\item 115200 bps
\end{DoxyItemize}

\subsubsection*{4.\-3 Enter Common Firmware Environment prompt}

With the serial connection open, power on the router and immediately start sending breaks (Control-\/\-C) to the device, if your luck holds you will be greeted with a C\-F\-E prompt. \begin{DoxyVerb}CFE>
\end{DoxyVerb}


If the router seems to start booting up, you can powercycle and try again.

\subsubsection*{4.\-4 Set I\-P address}

By default, the router will have a static I\-P address of 192.\-168.\-1.\-1. If you need to set a different address for your network, run one of the following commands\-: \begin{DoxyVerb}ifconfig eth0 -auto                      if you are using a DHCP server
ifconfig eth0 -addr=*.*.*.*              for a static IP address
\end{DoxyVerb}


\subsubsection*{4.\-5 Load image over T\-F\-T\-P}

On a computer that is network accessible from the router, start your T\-F\-T\-P server and place the xinu.\-boot image in the root directory that the server makes available.

Then, on the router type the command\-: \begin{DoxyVerb}CFE> boot -elf [TFTP server IP]:xinu.boot
\end{DoxyVerb}


If all has gone correctly the router you will be greeted with the Xinu Shell ({\ttfamily xsh\$}), which means you are now running Embedded Xinu!

\subsection*{5. \href{http://xinu.mscs.mu.edu/List_of_supported_platforms}{\tt Supported Platforms}}

\begin{TabularC}{3}
\hline
\rowcolor{lightgray}{\bf Platforms }&{\bf Status }&{\bf Comments  }\\\cline{1-3}
\href{http://www.linksys.com/servlet/Satellite?c=L_Product_C2&childpagename=US%2FLayout&cid=1149562300349&pagename=Linksys%2FCommon%2FVisitorWrapper}{\tt Linksys W\-R\-T54\-G v8} &Supported &Tested and running at the Embedded Xinu Lab. \\\cline{1-3}
\href{http://www.linksys.com/servlet/Satellite?c=L_Product_C2&childpagename=US%2FLayout&cid=1133202177241&pagename=Linksys%2FCommon%2FVisitorWrapper}{\tt Linksys W\-R\-T54\-G\-L} &Supported &This is our primary development platform, on which Xinu has been tested thoroughly. \\\cline{1-3}
\href{http://www.linksys.com/servlet/Satellite?c=L_Product_C2&childpagename=US%2FLayout&cid=1149562300349&pagename=Linksys%2FCommon%2FVisitorWrapper}{\tt Linksys W\-R\-T54\-G v4} &Probably Supported &The v4 is apparently the version on which W\-R\-T54\-G\-L is based, and so although the Embedded Xinu Lab has not explicitly tested it, it probably works. \\\cline{1-3}
\href{http://www.linksys.com/servlet/satellite?c=l_product_c2&childpagename=us%2flayout&cid=1162354643512&pagename=linksys%2fcommon%2fvisitorwrapper}{\tt Linksys W\-R\-T350\-N} &Under Development &Currently the synchronous U\-A\-R\-T Driver works. \\\cline{1-3}
\href{http://www.linksys.com/servlet/satellite?c=l_product_c2&childpagename=us%2flayout&cid=1149562300349&pagename=linksys%2fcommon%2fvisitorwrapper}{\tt Linksys W\-R\-T160\-N\-L} &Supported &Newer model of router. Full O/\-S teaching core functioning, including wired network interface. \\\cline{1-3}
\href{http://www.qemu.org/}{\tt mipsel-\/qemu} &Supported &Full O/\-S teaching core functioning, network support in progress. \\\cline{1-3}
\href{http://www.raspberrypi.org}{\tt Raspberry Pi} &Under Development &Core operating system including wired networking is functional. Some new features are still being worked on, and the full documentation (e.\-g. for a laboratory setup) hasn't been completed yet. \\\cline{1-3}
\href{http://www.raspberrypi.org}{\tt Raspberry Pi 3} &Under development &Work has been started remaking the low level systems to operate on the aarch64 architecture. \\\cline{1-3}
\end{TabularC}
\subsection*{6. Links}

\subsubsection*{Embedded Xinu documentation.}

Home of the most up to date documentation about new and upcoming projects.

\href{http://embedded-xinu.readthedocs.io/en/latest/}{\tt http\-://embedded-\/xinu.\-readthedocs.\-io/en/latest/}

\subsubsection*{The Embedded Xinu Wiki}

The home of the Embedded Xinu project

\href{http://xinu.mscs.mu.edu/}{\tt http\-://xinu.\-mscs.\-mu.\-edu/}

\subsubsection*{Dr. Brylow's Embedded Xinu Lab Infrastructure Page}

More information about the Embedded Xinu Lab at Marquette University

\href{http://www.mscs.mu.edu/~brylow/xinu/}{\tt http\-://www.\-mscs.\-mu.\-edu/$\sim$brylow/xinu/} 